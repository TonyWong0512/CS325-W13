\documentclass[12pt]{article}

\usepackage{verbatim}
\usepackage{amsmath, amssymb, amsthm}
\usepackage{qtree}
\usepackage{enumerate}
\usepackage{changepage}
\usepackage{tikz}
\usetikzlibrary{arrows}
\usetikzlibrary{automata}
\usepackage{subfigure}
\usepackage{pgfplots}

\usepackage[utf8]{inputenc}

\title{CS325 Winter 2013: HW 3}
\author{
    Daniel Reichert \\
    Trevor Bramwell \\
}
\date{\today}

\newcommand{\BigO}[1]{\ensuremath{O(#1)}}

% Big O: \BigO
% Big Omega: \Omega
% Big Theta: \Theta

% Examples:
%
%   $\BigO{n}$
%   $\Omega(n\log{n})$
%   $\Theta(\log{2n})$


\begin{document}
\maketitle
\section*{1}
\paragraph{Problem:}
Knapsack without repetitions. Consider the following knapsack problem:
The total weight limit W = 10 and \\
\begin{tabular}{ l | c | r }
    Item & Weight & Value \\ \hline
    1 & 6 & \$30 \\
    2 & 3 & \$14 \\
    3 & 4 & \$16 \\
    4 & 2 & \$9  \\ \hline
\end{tabular} \\
Solve this problem using the dynamic programming algorithm presented in class. Please show the two
dimensional table L(w, j) for w = 0, 1, ..., W and j = 1, 2, 3, 4.

\paragraph{Solution:}
\begin{proof}

\end{proof}

\section*{2}
\paragraph{Problem:}
Give a dynamic programming algorithm for solving the following problem.
Input: A list of n positive integers a1 , a2 , . . . , an and a number t.
Goal: Decide if some subset of the ai ’s add up to t. (You can use each ai at most once.)
The running time should be O(nt).
Consider the subproblem L(i, s), which returns the answer of “Does a subset of a1 , ..., ai sum up to s?”.
Below are some substeps that will help you develop your algorithm.
a What are the two options we have regarding item i toward answering the subproblem L(i, s)?
b For each of the option, how would it change the subproblem? More specifically, what happens to the
target sum and what happens to the set of available integars?
c Based on the answers to the previous two questions, write a recursive formula that expresses L(i, s)
using the solutions to smaller subproblems.
d Provide pseudocode for the dynamic programming algorithm that builds the solution table L(i, s) and
returns the correct answer to the final problem.
e Modify the pseudocode such that it will not only return the correct “yes”, “no” answer, but also return
the exact subset if the answer is “yes”.

\paragraph{Solution:}
\begin{proof}

\end{proof}

\section*{3}
\paragraph{6.2 from book:}
You are going on a long trip. You start on the road at mile post 0. Along the way there are n
hotels, at mile posts a1 < a2 < · · · < an , where each ai is measured from the starting point. The
only places you are allowed to stop are at these hotels, but you can choose which of the hotels
you stop at. You must stop at the final hotel (at distance an ), which is your destination.

You’d ideally like to travel 200 miles a day, but this may not be possible (depending on the spacing
of the hotels). If you travel x miles during a day, the penalty for that day is (200 − x)2 . You want
to plan your trip so as to minimize the total penalty—that is, the sum, over all travel days, of the
daily penalties.
Give an efficient algorithm that determines the optimal sequence of hotels at which to stop.
\paragraph{Solution:}
\begin{proof}

\end{proof}

\section*{4}
\paragraph{6.8 from book:}
Given two strings x = x1 x2 · · · xn and y = y1 y2 · · · ym , we wish to find the length of their longest
common substring, that is, the largest k for which there are in

\paragraph{Solution:}
\begin{proof}

\end{proof}
\end{document}
