\documentclass[12pt]{article}

\usepackage{amsmath}
\usepackage{amssymb}
\usepackage{enumerate}
\usepackage{changepage}

\title{CS325 Winter 2013: HW 2}
\author{
    Daniel Reichert \\
    Trevor Bramwell \\
    Lance Stringham
}
\date{\today}

\newcommand{\BigO}[1]{\ensuremath{O(#1)}}

% Big O: \BigO
% Big Omega: \Omega
% Big Theta: \Theta

% Examples:
%
%   $\BigO{n}$
%   $\Omega(n\log{n})$
%   $\Theta(\log{2n})$


\begin{document}
\maketitle

\section*{1}
\paragraph{Problem:}
Given two sorted arrays $a[1, ..., n]$ and $b[1, ..., n]$, given an $O(\log n)$
algorithm to find the median of their combined $2n$ elements. (Hint: use
divide and conquer).

\paragraph{Solution:}
\begin{verbatim}
med_a = a[n / 2]
med_b = b[n / 2]
i = 4
while (i < n)
    if (med_a > med_b)
        med_a = a[((i - 1) * n) / 2]
        med_b = b[n / i]
    else (med_a < med_b)
        med_a = a[n / i]
        med_b = b[((i - 1) * n) / 2]
    i = 2 * i
    if (i >= n)
        median = (med_a + med_b) / 2
        return median    
\end{verbatim}

\section*{2}
\paragraph{Problem:}
Prove the following statement by induction: In any full binary tree, the
number of leaves is exactly one more than the number of internal nodes.
Definition: A node is a leaf if it has no children; otherwise, it is an
internal node. A full binary tree is a binary tree whose node is either
a leaf node or an internal node with exactly two children.
\paragraph{Solution:}
Proof by induction.  
Base case: a full binary tree of height 1.
\Tree[.I[L]
	[L]]
This tree contains two leafs and one internal node, which satisfies the statement $L = I+1$, and proves our base case.

Inductive step.
	Assume: $L_k = I_k +1$ for $1 \lte k \lte n$
	Prove: $L_(k+1) = I_(k+1) +1$ 
By the definition of a full binary tree we know that any internal node may be cut such that the remaining two branches are also full binary trees.  This allows us to express any full binary tree as a set of recurisively defined full binary trees.

\Tree[.$I_1$[.$I_2$[.L]
	           [.L]]
	    [.$I_3$[.L]
	    	   [.$I_4$[.L]
	            	  [.L]]]]

In the above full binary tree, if the internal node $I_3$ was cut away from the tree to only be represented as in the tree below, the newly created tree is still a full binary tree.
\Tree[.$I_3$[.L]
	    [.$I_4$[.L]
		   [.L]]]
	    
			

\section*{3}
\paragraph{Problem:}
Interval scheduling. We are given a set of requests for using a
resource.  Each request i specifies a starting time s(i) and an end time
f (i). The resource can only accommodate one request at a time. If two
requests overlap in time, they are incompatible and cannot be both
fulfilled. The goal is to identify a maximum subset of compatible
requests. One possible greedy strategy is to select at each step the
request that is compatible with the maximum number of the remaining
requests. Will this greedy strategy lead to an optimal solution? If so,
provide a proof. If not, provide a counter example.
\paragraph{Solution:}
Proof by counter example. 
In the following counter example we will demonstrate that the greedy choice of maximal number of remaining requests is not optimal.

|--1--||--2--||--3--||--4--|
  |--5--| |--6--| |--7--|
  |--8--|         |--9--|
  |--10-|         |--11-|

In this example the optimal solution consists of choosing four intervals numbered 1-4.  Intervals 1 and 4 have seven remaining requests.  Intervals 2,3,5,7,8,9,10 an 11 have six remaining requests.  Interval 6 has a total of eight remaining requests.  Because interval 6 has the highest number of requests remaining, it would be selected first.  This first choice conflicts with intervals 2 and 3.  Thus only allowing a solution with a total of three intervals.  Thus, proof by contradiction.

\section*{4}
\paragraph{Problem:}
You and your friends are taking a long hiking trip of L miles, along
which there are n camping sites located at distances x1 , x2 , ..., xn
respectively from the start of the trip. You can hike at most d miles
per day, by the end of which you must stop and camp for the night. You
need make a valid trip plan that takes the minimum number of camping
stops. The plan should specify which camping sites to use, and it is
only valid if any two consecutive stops are no more than d miles apart.
Your friend proposed the following strategy: each time you come to a
camp site, check whether you can make it to the next site before the end
of the day (i.e., before finishing the d miles quota for the day. We
assume this can always be determined correctly), if so, keep hiking. If
not, stop for the night.  This is in fact a greedy algorithm, which
simply choose to hike as long as possible each day. Prove that this
greedy algorithm achieves the optimal solution, i.e, it uses the minimum
number of stops. (Hint: construct a proof that is similar to the
interval scheduling proof, which shows that the greedy algorithm stays
ahead.)

\paragraph{Solution:}

\end{document}
