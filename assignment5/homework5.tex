\documentclass[12pt]{article}

\usepackage{verbatim}
\usepackage{listings}
\usepackage{amsmath, amssymb, amsthm}
\usepackage{qtree}
\usepackage{enumerate}
\usepackage{changepage}
\usepackage{tikz}
\usetikzlibrary{arrows}
\usetikzlibrary{automata}
\usepackage{subfigure}
\usepackage{pgfplots}

\usepackage{enumitem}
\setdescription{leftmargin=\parindent,labelindent=\parindent}
\setlist[enumerate,1]{label=\alph*}

\usepackage[utf8]{inputenc}

\title{CS325 Winter 2013: HW 5}
\author{
    Daniel Reichert \\
    Trevor Bramwell \\
}
\date{\today}

\newcommand{\BigO}[1]{\ensuremath{O(#1)}}

% Big O: \BigO
% Big Omega: \Omega
% Big Theta: \Theta

% Examples:
%
%   $\BigO{n}$
%   $\Omega(n\log{n})$
%   $\Theta(\log{2n})$


\begin{document}
\maketitle
\section*{1}
\paragraph{Problem:}
We will review proof by induction in this problem. Specifically, consider the
stoogesort algorithm for sorting and use proof by induction to show that this
algorithm correctly sorts an given array into increasing order.
\begin{lstlisting}[mathescape]
STOOGESORT(A[0...n-1])
    if n = 2 and A[0] > A[1]
        swap A[0] and A[1]
    else if n > 2
        k = 2n/3
        STOOGESORT(A[0...k-1])
        STOOGESORT(A[n-k...n-1])
        STOOGESORT(A[0...k-1])
\end{lstlisting}
\paragraph{Proof By induction:}
\begin{proof}
Base case: $n = 2$.
If $n=2$ and $A[0] = 2$ and $A[1] = 1$, then the algorithm if statement
will evaluate to true.  This will cause $A[0]$ and $A[1]$ to be swapped,
and the array is now sorted in increasing order proving our base case.
\\
Inductive Step:
Assume Stoogesort is true for $2...K-1$.
Prove Stoogesort is true for $K$.
\\
When stoogesort has a input greater than 2, it calls stoogesort($2K/3$).
Since $2K/3$ is within our Inductive Assumption, we know that
stoogesort($2K/3$) is true.  Since stoogesort always relies on being
able to successfully sort a subset that is within the scope of our
inductive assumption, by the principal of mathematical inducation,
stooge sort is true.
\end{proof}

\section*{2}
\paragraph{6.17 from book:}
Given an unlimited supply of coins of denominations $x_1 , x_2 , \dots ,
x_n$, we wish to make change for a value $v$; that is, we wish to find a
set of coins whose total value is $v$. This might not be possible: for
instance, if the denominations are 5 and 10 then we can make change for
15 but not for 12.  Give an $\BigO{nv}$ dynamic-programming algorithm
for the following problem.  Input: $x_1 , \dots , x_n ; v$.  Question:
Is it possible to make change for $v$ using coins of denominations $x_1
, \dots , x_n$ ?
\paragraph{Solution:}
\begin{lstlisting}[mathescape]
MAKE_CHANGE(A, v):
    coins[$i$] := 0 for 0 to $v$
    for $i$ from 1 to $v$:
        for $j$ from 1 to $A$:
            if $i > j$:
                min := coins[$i-j$] + 1
                if min < coins[$i$]:
                    coins[$i$] := min
    if coins[$i$] != 0:
	return True
    return False
\end{lstlisting}

\section*{3}
\paragraph{Problem:}
Additional question on 6.17. Show how one can reduce the problem specified by
6.17 into a knapsack problem (with repetition). Reducing to knapsack problem
means that for any instance of the coin-change problem specified in 6.17, we
can turn it into a knapsack problem, and apply an algorithm for knapsack, and
use its solution to decide the solution for the original coin change problem.
\paragraph{Solution:}

The problem in 6.17 is a coin-change problem. The coin change problem
can be reduced to a knapsack problem. To do so, let the denominations of
coins be the values of items to be placed in the knapsack and let the
total value of items be equal to the weight. The weight of an individual
element is ignored.

\section*{4}
\paragraph{6.26 from book:}
Sequence alignment. When a new gene is discovered, a standard approach to
understanding its function is to look through a database of known genes and
find close matches. The closeness of two genes is measured by the extent to 
which they are aligned. To formalize this, think of a gene as being a long
string over an alphabet $\sum = \{A, C, G, T\}$. Consider two genes (strings)
$x = AT GCC$ and $y = T ACGCA$. An alignment of $x$ and $y$ is a way of matching up
these two strings by writing them in columns, for instance:

\begin{center}
\begin{tabular}{c c c c c c c}
-- &A &T &-- &G &C &C \\
T &A &-- &C &G &C &A
\end{tabular}
\end{center}

Here the ``-'' indicates a ``gap.'' The characters of each string must appear
in order, and each column must contain a character from at least one of 
the strings. The score of an alignment is specified by a scoring matrix
$\delta$
of size $(|\sum| + 1) * (|\sum| + 1)$, where the extra row and column are to
accommodate gaps. For instance the preceding alignment has the following score:
$$\delta(-, T ) + \delta(A, A) + \delta(T, -) + \delta(-, C) + \delta(G,
G) + \delta(C, C) + \delta(C, A)$$

Give a dynamic programming algorithm that takes as input two strings
$x[1 \dots n]$ and $y[1 \dots m]$ and a scoring matrix $\delta$, and returns
the highest-scoring alignment. The running time should be $\BigO{mn}$.

\paragraph{Solution:}
This solution builds on the dynamic programming algorithm for edit
distance. Instead of finding a minimum edit, we find a maximum
alignment score.
\begin{lstlisting}[mathescape]
$n$ is the length of $A$
$m$ is the length of $B$
$\delta$ is a scoring function of size $(|\sum| + 1) * (|\sum| + 1)$

SEQ_ALIGN($A$, $B$, $\delta$):
    score[$i$, 0] = 0, for $0,...,n$
    score[0, $j$] = 0, for $0,...,m$
    for $i$ from 1 to $n$:
        for $j$ from 1 to $m$:
            ins = score[$i-1$, $j$] + 1
            del = score[$i$, $j-1$] + 1
            sub = score[$i-1$, $j-1$] + $\delta$($A_i$, $B_j$)
            score[$i$, $j$] = max{ins, del, sub}
    return score[$n$, $m$]
\end{lstlisting}
\end{document}
