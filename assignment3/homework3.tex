\documentclass[12pt]{article}

\usepackage{verbatim}
\usepackage{amsmath, amssymb, amsthm}
\usepackage{qtree}
\usepackage{enumerate}
\usepackage{changepage}
\usepackage{TikZ}

\title{CS325 Winter 2013: HW 3}
\author{
    Daniel Reichert \\
    Trevor Bramwell \\
    Lance Stringham
}
\date{\today}

\newcommand{\BigO}[1]{\ensuremath{O(#1)}}

% Big O: \BigO
% Big Omega: \Omega
% Big Theta: \Theta

% Examples:
%
%   $\BigO{n}$
%   $\Omega(n\log{n})$
%   $\Theta(\log{2n})$


\begin{document}
\maketitle

\section{4.1}
\paragraph{Problem:}
Suppose Dijkstra’s algorithm is run on the following graph, starting at node A.
\begin{tikzpicture}[->,>=stealth',shorten >=1pt,auto,node distance=3cm,
  thick,main node/.style={circle,fill=blue!20,draw,font=\sffamily\Large\bfseries}]

  \node[main node] (a) {a};
  \node[main node] (b) [right of=a]{b};
  \node[main node] (c) [right of=b]{c};
  \node[main node] (d) [right of=c]{d};
  \node[main node] (e) [below of=a]{e};
  \node[main node] (f) [below of=b]{f};
  \node[main node] (g) [below of=c]{g};
  \node[main node] (h) [below of=d]{h};
 
  \path[every node/.style={font=\sffamily\small}]
    (a) edge node [left] {1} (b)
        edge node [down] {4} (e)
        edge node {8} (f)
    (b) edge node [left] {2} (c)
        edge node [down] {6} (f)
        edge node {6} (g)
    (c) edge node [left] {1} (d)
        edge node [down] {2} (g)
    (d) edge node [down] {4} (h)
        edge node {1} (g)
    (e) edge node [left] {5} (f)
    (g) edge node [right] {1} (f)
	edge node [left] {1} (h)
\end{tikzpicture}
(a) Draw a table showing the intermediate distance values of all the nodes at each iteration of
the algorithm.
(b) Show the final shortest-path tree.

\paragraph{Solution:}
\begin{proof}
\end{proof}

\section{5.2}
\paragraph{Problem:}
Suppose we want to find the minimum spanning tree of the following graph.

\begin{tikzpicture}[->,>=stealth',shorten >=1pt,auto,node distance=3cm,
  thick,main node/.style={circle,fill=blue!20,draw,font=\sffamily\Large\bfseries}]
  \node[main node] (a) {a};
  \node[main node] (b) [right of=a]{b};
  \node[main node] (c) [right of=b]{c};
  \node[main node] (d) [right of=c]{d};
  \node[main node] (e) [below of=a]{e};
  \node[main node] (f) [below of=b]{f};
  \node[main node] (g) [below of=c]{g};
  \node[main node] (h) [below of=d]{h};
  \tikzset{LabelStyle/.style =   {draw,
				  fill           = yellow,
				  text           = red}}
  \path[every node/.style={font=\sffamily\small}]
  \draw[EdgeStyle](a) to node[LabelStyle]{1} (b);
  \draw[EdgeStyle](a) to node[LabelStyle]{8} (f);
  \draw[EdgeStyle](a) to node[LabelStyle]{4} (e);
  \draw[EdgeStyle](b) to node[LabelStyle]{2} (c);
  \draw[EdgeStyle](b) to node[LabelStyle]{6} (g);
  \draw[EdgeStyle](b) to node[LabelStyle]{6} (f);
  \draw[EdgeStyle](c) to node[LabelStyle]{3} (d);
  \draw[EdgeStyle](c) to node[LabelStyle]{2} (g);
  \draw[EdgeStyle](d) to node[LabelStyle]{1} (g);
  \draw[EdgeStyle](d) to node[LabelStyle]{4} (h);
  \draw[EdgeStyle](e) to node[LabelStyle]{5} (f);
  \draw[EdgeStyle](f) to node[LabelStyle]{1} (g);
  \draw[EdgeStyle](g) to node[LabelStyle]{1} (h);
\end{tikzpicture}
(a) Run Prim’s algorithm; whenever there is a choice of nodes, always use alphabetic ordering
(e.g., start from node A). Draw a table showing the intermediate values of the cost array.
(b) Run Kruskal’s algorithm on the same graph. Show how the disjoint-sets data structure
looks at every intermediate stage (including the structure of the directed trees), assuming
path compression is used.

\paragraph{Solution:}
\begin{proof}
\end{proof}


\section{5.5}
\paragraph{Problem:}

Consider an undirected graph $G = (V, E)$ with nonnegative edge weights $w_e \gte 0$. Suppose that
you have computed a minimum spanning tree of G, and that you have also computed shortest
paths to all nodes from a particular node $s ∈ V$.
Now suppose each edge weight is increased by 1: the new weights are $w_e = w_e + 1$.
(a) Does the minimum spanning tree change? Give an example where it changes or prove it
cannot change.
(b) Do the shortest paths change? Give an example where they change or prove they cannot
change.
\paragraph{Solution:}
\begin{proof}
Part A:  Proof by counter example.
Consider the graph:

\begin{tikzpicture}[->,>=stealth',shorten >=1pt,auto,node distance=3cm,
  thick,main node/.style={circle,fill=blue!20,draw,font=\sffamily\Large\bfseries}]

  \node[main node] (a) {a};
  \node[main node] (b) [right of=a]{b};
  \node[main node] (c) [right of=b]{c};
  \node[main node] (d) [right of=c]{d};
 
  \path[every node/.style={font=\sffamily\small}]
    (a) edge node [left] {1} (b)
        edge node [down] {4} (d)
    (b) edge node [left] {1} (c)
    (c) edge node [left] {1} (d)
\end{tikzpicture}
In this graph the shortest path from A to D is from A to B to C to D with a weight of 3.
Now consider the same graph, but with all of the edges having a weight increased by 1.

\begin{tikzpicture}[->,>=stealth',shorten >=1pt,auto,node distance=3cm,
  thick,main node/.style={circle,fill=blue!20,draw,font=\sffamily\Large\bfseries}]

  \node[main node] (a) {a};
  \node[main node] (b) [right of=a]{b};
  \node[main node] (c) [right of=b]{c};
  \node[main node] (d) [right of=c]{d};
 
  \path[every node/.style={font=\sffamily\small}]
    (a) edge node [left] {2} (b)
        edge node [down] {5} (d)
    (b) edge node [left] {2} (c)
    (c) edge node [left] {2} (d)
\end{tikzpicture}
The previously shortest path now has a weight of 6, where the direct route from A to D only has a weight of 5.  Thus the shortest path has changed and the counter example is proved.  \qed
\end{proof}
\begin{proof}
Part B:
\end{proof}



\section{5.7}
\paragraph{Problem:}
Show how to find the maximum spanning tree of a graph, that is, the spanning tree of largest
total weight.
\paragraph{Solution:}
\begin{proof}
\end{proof}

\section{5}
\paragraph{Problem:}
Consider the Change Problem in Austria. The input to this problem is
an integer L. The output should be the minimum cardinality collection
of coins required to make L shillings of change (that is, you want to use
as few coins as possible). In Austria the coins are worth 1, 5, 10, 20, 25,
50 Shillings. Assume that you have an unlimited number of coins of each
type. Formally prove or disprove that the greedy algorithm (that takes as
many coins as possible from the highest denominations) correctly solves
the Change Problem. So for example, to make change for 234 Shillings the
greedy algorithms would take four 50 shilling coins, one 25 shilling coin,
one 5 shilling coin, and four 1 shilling coins.
\paragraph{Solution:}
\begin{proof}
Proof by counter example.
When the Shilling total is equal to 40, the optimal solution would use two 20 shilling coins, so $n=2$ where $n$ is the number of coins.  According to the greedy method of taking as many coins as possible from the highest denomination available, a shilling total of 40 would have change made with one 25 shilling coin, one 15 shilling coin, and one 5 shillingcoin making $n=3$ which is not optimal.
\end{proof}

\section{6}
\paragraph{Problem:}
Consider a long quiet country road with houses scatter very sparsely along
it (We can picture the road as a long line segment). You want to place cell
phone base stations at certain points along the road, so that every house
is within four miles of one of the base stations.
Give an efficient algorithm that achieves this goal, using as few stations
as possible. Show that the algorithm achieve the optimal solution using
the “stay ahead” argument.
\paragraph{Solution:}
\begin{proof}

\end{proof}
\end{document}
