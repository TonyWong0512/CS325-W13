\documentclass[12pt]{article}

\usepackage{amsmath, amssymb, amsthm}
\usepackage{enumerate}
\usepackage{changepage}
\usepackage{pgfplots, pgfplotstable}

\title{CS325 Winter 2013: Implementation 3}
\author{
    Daniel Reichert \\
    Trevor Bramwell
}
\date{\today}

\begin{document}
\maketitle

\section*{Approach Overview}

In our approach to the TSP we used a few different methods.  The first step is
to run k-means to produce $k$ distinct subsets of the cities that are
localized.  This produces neighborhoods that effectively divides the program
into small enough parts such that an optimal solution is more easily found.  It
is also a prerequisite to allowing a parallelized algorithm.  

Next, for each neighborhood we ran the greedy nearest neighbor algorithm to
produce an initial neighborhood tour which we would later optimize.

After acquiring the optimal tour that nearest neighbor can produce, we run
3-opt on the tour to exhaustion. According to Johnson and McGeoch [table 10]
3-opt can produce results that are within 2 percent of the optimal tour of the
graph.

Once 3-opt is run on the tour, it is necessary to connect each of the
neighborhoods together in order to create a single tour of all of the cities.
One at a time the neighborhoods are merged together until a complete tour of
all cities is established.

\begin{Steps in Detail}

\section*{K-means}

\section*{Nearest neighbor}

Nearest neighbor starts at a given node and then finds the closest node to
itself and visits it. As each node is visited is subsequently removed from the
pool of nodes that are considered for the next node. Before further optimizing
the tour that is found by nearest neighbor, we run the nearest neighbor process
repeatedly. Because the nearest neighbor algorithm is cheap to run, we are able
to find the minimal length tour that it can produce by considering every
possible node as the starting point. 

\section*{K-opt}

\section*{connecting neighborhoods}

The process of connecting the neighborhoods together involves figuring out what
order the neighborhoods should be connected in. This is accomplished by
repeating the nearest neighbor and 3-opt process, but instead of running it on
a subgraph of the problem, the neighborhoods are treated as a unit.  In order
to treat them as a single unit, we find the center of each neighborhood by
finding the average $x$ and $y$ value and considering that for a node.  With
the order in which the neighborhoods should be glued together established, a
method that is very similar to the swap function used in 3-opt is used to merge
each neighborhood tour into a single tour.

\end{Steps in detail}

http://www2.research.att.com/~dsj/papers/TSPchapter.pdf


\end{document}
