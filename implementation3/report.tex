\documentclass[12pt]{article}

\usepackage{amsmath, amssymb, amsthm}
\usepackage{enumerate}
\usepackage{changepage}
\usepackage{fullpage}
\usepackage{pgfplots, pgfplotstable}
\usepackage{url}

\title{CS325 Winter 2013: Implementation 3 Report}
\author{
    Daniel Reichert \\
    Trevor Bramwell
}
\date{\today}

\begin{document}
\maketitle

\section*{Approach Overview}

In our approach to the TSP we used a few different methods.  The first step is
to run k-means to produce $k$ distinct subsets of the cities that are
localized.  This produces neighborhoods that effectively divides the program
into small enough parts such that an optimal solution is more easily found.  It
is also a prerequisite to allowing a parallelized algorithm.  

Next, for each neighborhood we ran the greedy nearest neighbor algorithm to
produce an initial neighborhood tour which we would later optimize.

After acquiring the optimal tour that nearest neighbor can produce, we run
2-opt on the tour to exhaustion. According to Johnson and McGeoch [table 10]
2-opt can produce results that are within 5 percent of the optimal tour of the
graph.

Once 2-opt is run on the tour, it is necessary to connect each of the
neighborhoods together in order to create a single tour of all of the cities.
One at a time the neighborhoods are merged together until a complete tour of
all cities is established.

\subsection*{Steps in Detail}

\paragraph{K-means++}

K-means++ needs to be passed a graph and $k$, where $k$ is the number of
clusters that will be created. It starts by choosing a node a random
from the graph. Once a node is selected it then computes the node
furthest from itself, then the node furthest from the first and second.
This continues till k nodes are chosen. Then each node in the graph
(with the exception of the nodes chosen for $k$) associates itself with
the nearest cluster.

\paragraph{Nearest neighbor}

Nearest neighbor starts at a given node and then finds the closest node to
itself and visits it. As each node is visited is subsequently removed from the
pool of nodes that are considered for the next node. Before further optimizing
the tour that is found by nearest neighbor, we run the nearest neighbor process
repeatedly. Because the nearest neighbor algorithm is cheap to run, we are able
to find the minimal length tour that it can produce by considering every
possible node as the starting point. 

\paragraph{2-OPT}

2-OPT is a local search heuristic that replaces two edges in a TSP graph
with two cheaper edges. It can either be ran once, or until the cheapest
local option is found. The 2-OPT we implemented was the exaustive search
for the cheapest option.

\paragraph{connecting neighborhoods}

The process of connecting the neighborhoods together involves figuring out what
order the neighborhoods should be connected in. This is accomplished by
repeating the nearest neighbor and 3-opt process, but instead of running it on
a subgraph of the problem, the neighborhoods are treated as a unit.  In order
to treat them as a single unit, we find the center of each neighborhood by
finding the average $x$ and $y$ value and considering that for a node.  With
the order in which the neighborhoods should be glued together established, a
method that is very similar to the swap function used in 3-opt is used to merge
each neighborhood tour into a single tour.


\section*{Implementation}

While our approach had great promise, its complexity proved to be too much
given our allowed time frame to implement it.  There were a small number of
bugs and false assumptions which prevented us from using each piece in the
solution that we submitted.

The most important false assumption that we made was that each point would be
distinct.  Because in our minds no graph that someone would want to apply a TSP
algorithm to would have duplicate points, we did not consider that possibility
in our algorithms.  However some of the test and example input contained
duplicate points, and as such broke our algorithms in unexpected ways.  This
took an inordinate amount of time to debug, and prevented us from using all of
the functions that we had implemented all at once.

There were also some bugs in our algorithms that prevented us from utilizing 

\section*{Resources Used}

\emph{The Traveling Salesman Problem: A Case Study in Local Optimization}.  By
David S. Johnson and Lyle A. McGeoch. November 20, 1995.
\url{http://www2.research.att.com/~dsj/papers/TSPchapter.pdf}\\
\\

\emph{The Traveling Salesman Problem (TSP)}.  By Rahul Simha. The George
Washington University
\url{http://www.seas.gwu.edu/~simhaweb/champalg/tsp/tsp.html}\\
\\

\emph{Pattern Recognition and Machine Learning}. By Christopher M. Bishop.
p424-450. 2006.\\
\\

\emph{Algorithms}. By Dasgupta, Papadimitriou, and Vazirani. p283-305. July 18,
2006.  \url{http://www.cs.berkeley.edu/%7Evazirani/algorithms.html}

http://en.wikipedia.org/wiki/2-opt

\end{document}
