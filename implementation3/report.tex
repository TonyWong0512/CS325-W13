\documentclass[12pt]{article}

\usepackage{amsmath, amssymb, amsthm}
\usepackage{enumerate}
\usepackage{changepage}
\usepackage{pgfplots, pgfplotstable}

\title{CS325 Winter 2013: Implementation 3}
\author{
    Daniel Reichert \\
    Trevor Bramwell
}
\date{\today}

\begin{document}
\maketitle

\section*{Approach}

In our approach to the TSP we used a few different methods.  The first step is
to run k-means to produce $k$ distinct subsets of the cities that are
localized.  This produces neighborhoods that effectively divides the program
into small enough parts such that an optimal solution is more easily found.  It
is also a prerequisite to allowing a parallelized algorithm.  

Next, for each neighborhood we ran the greedy nearest neighbor algorithm to
produce an initial neighborhood tour which we would later optimize. However
before optimizing it we maximized the benefit of the nearest neighbor
algorithm. Because nearest neighbor is cheap to run, we were able to run it
repeatedly. Each time we choose a different starting node. This allowed us to
iterate across all possible starting nodes and pick the optimal starting node
before moving on to any further steps.

After acquiring the optimal tour that nearest neighbor can produce, we run
3-opt on the tour to exhaustion. According to Johnson and McGeoch [table 10]
3-opt can produce results that are within 2 percent of the optimal tour of the
graph.

Once 3-opt is run on the tour, it is necessary to connect each of the
neighborhoods together in order to create a single tour of all of the cities.
This was accomplished by repeating the nearest neighbor and 3-opt, but instead
of running it on a subgraph of the problem, the neighborhoods were treated as a
unit.  In order to treat them as a single unit, we found the center of each
neighborhood by finding the average $x$ and $y$ value and considering that for
a node.  With the order in which the neighborhoods should be glued together
established, a method that is very similar to the swap function used in 3-opt
is used to merge each neighborhood of nodes into a single tour.



http://www2.research.att.com/~dsj/papers/TSPchapter.pdf


\end{document}
