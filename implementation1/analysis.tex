\documentclass[12pt]{article}

\usepackage{amsmath, amssymb, amsthm}
\usepackage{enumerate}
\usepackage{changepage}
\usepackage{pgfplots}

\title{CS325 Winter 2013: Implementation 1}
\author{
    Daniel Reichert \\
    Trevor Bramwell
}
\date{\today}

\begin{document}
\maketitle

\section*{Asymptotic Analysis}
    \begin{enumerate}

    \item In algorithm 1 the problem is not branched into any subproblems.
          With the nested for loops there is a comparison made between
          every element in the input, leading to an intuitive asymptotic
          complexity of $O(n^2)$

	\item In algorithm 2 the problem is branched into $2$ subproblems
          for size $n/2$ at each level.

    \item In algorithm 3 the problem is branched into $2$ subproblems of size
          $n/2$ at each level.  The depth of the problem is $log_2 of n$
          and the width is $n^log_2 2$.  From the master theorem we know
          that $A/B^D$ determines the run time complexity.  Since this
          case is $2/2^1 = 1$, we know that the Asymptotic complexity is
          $O(n log n)$.

    \end{enumerate}

\section*{Testing}

\begin{center}
\begin{tabular}{|c|c|}
\hline
verify.txt & test\_input.txt \\ \hline
9670  & 249310 \\ \hline
10567 & 252709 \\ \hline 
9282  & 253719 \\ \hline 
9269  & 249315 \\ \hline 
9675  & 247789 \\ \hline 
10378 & 254833 \\ \hline 
9911  & 239844 \\ \hline 
9790  & 257527 \\ \hline 
9580  & 241669 \\ \hline 
9965  & 255628 \\
\hline
\end{tabular}
\end{center}

\section*{Extrapolation and Interpretation}


\end{document}
