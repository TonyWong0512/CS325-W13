\documentclass[12pt]{article}

\usepackage{verbatim}
\usepackage{listings}
\usepackage{amsmath, amssymb, amsthm}
\usepackage{qtree}
\usepackage{enumerate}
\usepackage{changepage}
\usepackage{tikz}
\usetikzlibrary{arrows}
\usetikzlibrary{automata}
\usepackage{subfigure}
\usepackage{pgfplots}

\usepackage{enumitem}
\setdescription{leftmargin=\parindent,labelindent=\parindent}
\setlist[enumerate,1]{label=(\alph*)}

\usepackage[utf8]{inputenc}

\title{CS325 Winter 2013: HW 6}
\author{
    Daniel Reichert \\
    Trevor Bramwell \\
}
\date{\today}

\newcommand{\BigO}[1]{\ensuremath{O(#1)}}

% Big O: \BigO
% Big Omega: \Omega
% Big Theta: \Theta

% Examples:
%
%   $\BigO{n}$
%   $\Omega(n\log{n})$
%   $\Theta(\log{2n})$


\begin{document}
\maketitle
\section*{1}
\paragraph{Problem:}
Below is a definition of the graph isomorphism problem.\\
\begin{description}
\item[Input:]
        two graphs, $G_1 = (V_1 , E_1)$, and $G_2 = (V_2 , E_2)$.
\item[Question:] Can the nodes of $G_1$ be renamed s.t. $G_1$ becomes $G_2$ ?\\
        In other words, is there a one-to-one function $f : V_1 \rightarrow V_2$
        such that for any edge $(x, y) \in E_1$ if only if
        $(f (x), f (y)) \in E_2$. Show that the graph Isomorphism problem is in NP.
\end{description}

\paragraph{Solution}
    To show that Graph Isomorphism (GI) is in NP we will show that 3-SAT
reduces to GI. Each literal in 3-SAT is a node in a graph, and each
clause is a group of nodes. Moving a clause from the last position to
the first will create a new graph isomorphic to the original, and it
will take NP to compare each node.

\section*{2}
\paragraph{8.4}
Consider the CLIQUE problem restricted to graphs in which every vertex has degree at most 3.
Call this problem CLIQUE-3.
\begin{enumerate}
\item Prove that CLIQUE -3 is in NP.
\item What is wrong with the following proof of NP-completeness for CLIQUE -3?
        We know that the CLIQUE problem in general graphs is NP-complete, so it is enough to
        present a reduction from CLIQUE -3 to CLIQUE. Given a graph G with
        vertices of degree $\le 3$, and a parameter g, the reduction
        leaves the graph and the parameter unchanged: clearly the output
        of the reduction is a possible input for the CLIQUE problem.
        Furthermore, the answer to both problems is identical. This
        proves the correctness of the reduction and, therefore, the
        NP-completeness of CLIQUE -3.
\item It is true that the VERTEX COVER problem remains NP-complete even when restricted to
    graphs in which every vertex has degree at most 3. Call this problem
    VC -3. What is wrong with the following proof of NP-completeness for
    CLIQUE -3?  We present a reduction from VC -3 to CLIQUE -3. Given a
    graph G = (V, E) with node degrees bounded by 3, and a parameter b,
    we create an instance of CLIQUE -3 by leaving the graph unchanged
    and switching the parameter to $|V| - b$. Now, a subset $C \subset
    V$ is a vertex cover in G if and only if the complementary set $V -
    C$ is a clique in $G$. Therefore G has a vertex cover of size $\le
    b$ if and only if it has a clique of size $\ge |V| - b$. This proves
    the correctness of the reduction and, consequently, the
    NP-completeness of CLIQUE-3.
\item Describe an \BigO{$(|V|4)$} algorithm for CLIQUE-3.
\end{enumerate}

\paragraph{Solution:}
\begin{enumerate}
\item A) Given a clique in a graph, all that is required to prove the certificate
    is that there exists and edge between every pair of verticies, which is easily
    done in polynomial time.
\item B) The reduction is done in the wrong order.  To prove that clique-3 is at
    least as hard as clique, clique must be reduced to clique-3.
\item C) The statement "Now, a subset C ⊆ V is a vertex cover in G if and only
    if the complementary set V − C is a clique in G" is incorrect.  This would
    be correct if independent set was used instead of clique.
\item D) With the restriction that every vertex in our clique problem is limited
    to a degree of at most 3, we know that there cannot exist a clique that is
    larger than 4 verticies.  To assume that more than 4 verticies can exist in
    a clique-3 graph breaks the definition of 3-clique and its requirement of
    being fully connected to every other vertex in the graph.  Therefore, we
    know that no solution exists for magnitudes of greater than 4.  With that
    knowledge we can restrict our search of solutions to $O(|v|^{4})$ time.
\end{enumerate}


\section*{3}
\paragraph{Problem:}
In the HITTING SET problem, we are given a family of sets {S1 , S2 , . . . , Sn } and a budget b, and
we wish to find a set H of size $\le b$ which intersects every $S_i$ , if such an H exists. In other words,
we want $H \cap S_i = \oslash$ for all i.
Show that
HITTING SET
is NP-complete.

\paragraph{Solution:}
This can be thought of as a vertex-cover.  Given a graph $G=(V,E)$ we can
describe each edge as a two element set that consists of the two vertices it
connects.  In other words, $e_i = (u,w)$ where $e$ is an edge and $(u,w)$ are
verticies.  This builds a family of sets from our given graph G, which allows
us to intuitively reduce this to an NP graph problem.  With this method of
thinking finding a vertex cover uses the same method as the hitting set problem.


\section*{4}
\paragraph{6.26 from book:}
Proving NP-completeness by generalization. For each of the problems below, prove that it is NP-
complete by showing that it is a generalization of some NP-complete problem we have seen in
this chapter.
\begin{enumerate}
\item  SUBGRAPH ISOMORPHISM: Given as input two undirected graphs G and H, determine
        whether G is a subgraph of H (that is, whether by deleting certain vertices and edges of H
        we obtain a graph that is, up to renaming of vertices, identical to G), and if so, return the
        corresponding mapping of V (G) into V (H).

\item LONGEST PATH :
        Given a graph G and an integer g, find in G a simple path of length g.

\item  DENSE SUBGRAPH: Given a graph and two integers a and b, find a set of a vertices of G
        such that there are at least b edges between them.

\item SPARSE SUBGRAPH: Given a graph and two integers a and b, find a set of a vertices of G
        such that there are at most b edges between them.
\end{enumerate}

\paragraph{Solution:}
\begin{enumerate}

\item A) The subgraph isomorphism can be reduced to the clique problem.

\item B) The longest path problem can be reduced to the rudrata path problem.

\item D) The dense subgraph problem can be reduced to the clique problem.

\item E) The sparse subgraph problem can be reduced to the independent set problem.

\end{enumerate}

\end{document}
