\documentclass[12pt]{article}

\usepackage{verbatim}
\usepackage{listings}
\usepackage{amsmath, amssymb, amsthm}
\usepackage{qtree}
\usepackage{enumerate}
\usepackage{changepage}
\usepackage{tikz}
\usetikzlibrary{arrows}
\usetikzlibrary{automata}
\usepackage{subfigure}
\usepackage{pgfplots}

\usepackage{enumitem}
\setdescription{leftmargin=\parindent,labelindent=\parindent}
\setlist[enumerate,1]{label=\alph*}

\usepackage[utf8]{inputenc}

\title{CS325 Winter 2013: HW 6}
\author{
    Daniel Reichert \\
    Trevor Bramwell \\
}
\date{\today}

\newcommand{\BigO}[1]{\ensuremath{O(#1)}}

% Big O: \BigO
% Big Omega: \Omega
% Big Theta: \Theta

% Examples:
%
%   $\BigO{n}$
%   $\Omega(n\log{n})$
%   $\Theta(\log{2n})$


\begin{document}
\maketitle
\section*{1}
\paragraph{Problem:}
Below is a definition of the graph isomorphism problem.\\
Input: two graphs, $G1 = (V1 , E1 )$, and $G2 = (V2 , E2 )$.
Question: Can the nodes of $G1$ be renamed s.t. $G1$ becomes $G2$ ?\\
In other words, is there a one-to-one function $f : V1 → V2$ such that for any edge $(x, y) ∈ E1$ if only if
$(f (x), f (y)) ∈ E2$. Show that the graph Isomorphism problem is in NP.

\paragraph{Solution}

\section*{2}
\paragraph{8.4}
Consider the CLIQUE problem restricted to graphs in which every vertex has degree at most 3.
Call this problem CLIQUE -3.
(a) Prove that
CLIQUE -3
is in NP.
(b) What is wrong with the following proof of NP-completeness for CLIQUE -3?
We know that the CLIQUE problem in general graphs is NP-complete, so it is enough to
present a reduction from CLIQUE -3 to CLIQUE. Given a graph G with vertices of degree ≤ 3,
and a parameter g, the reduction leaves the graph and the parameter unchanged: clearly
the output of the reduction is a possible input for the CLIQUE problem. Furthermore, the
answer to both problems is identical. This proves the correctness of the reduction and,
therefore, the NP-completeness of CLIQUE -3.
(c) It is true that the VERTEX COVER problem remains NP-complete even when restricted to
graphs in which every vertex has degree at most 3. Call this problem VC -3. What is wrong
with the following proof of NP-completeness for CLIQUE -3?
We present a reduction from VC -3 to CLIQUE -3. Given a graph G = (V, E) with node degrees
bounded by 3, and a parameter b, we create an instance of CLIQUE -3 by leaving the graph
unchanged and switching the parameter to |V | − b. Now, a subset C ⊆ V is a vertex cover
in G if and only if the complementary set V − C is a clique in G. Therefore G has a vertex
cover of size ≤ b if and only if it has a clique of size ≥ |V | − b. This proves the correctness of
the reduction and, consequently, the NP-completeness of CLIQUE -3.
(d) Describe an O(|V |4 ) algorithm for CLIQUE -3.

\paragraph{Solution:}

\section*{3}
\paragraph{Problem:}
In the HITTING SET problem, we are given a family of sets {S1 , S2 , . . . , Sn } and a budget b, and
we wish to find a set H of size ≤ b which intersects every Si , if such an H exists. In other words,
we want H ∩ Si = ∅ for all i.
Show that
HITTING SET
is NP-complete.

\paragraph{Solution:}

\section*{4}
\paragraph{6.26 from book:}
Proving NP-completeness by generalization. For each of the problems below, prove that it is NP-
complete by showing that it is a generalization of some NP-complete problem we have seen in
this chapter.
(a) S UBGRAPH ISOMORPHISM: Given as input two undirected graphs G and H, determine
whether G is a subgraph of H (that is, whether by deleting certain vertices and edges of H
we obtain a graph that is, up to renaming of vertices, identical to G), and if so, return the
corresponding mapping of V (G) into V (H).
(b) L ONGEST
PATH :
Given a graph G and an integer g, find in G a simple path of length g.
(d) D ENSE SUBGRAPH: Given a graph and two integers a and b, find a set of a vertices of G
such that there are at least b edges between them.
(e) S PARSE SUBGRAPH: Given a graph and two integers a and b, find a set of a vertices of G
such that there are at most b edges between them.

\paragraph{Solution:}
\end{document}
